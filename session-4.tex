%! Author = verwoerd
%! Date = 16-9-2023
% Preamble
\documentclass[11pt,pdf, aspectratio=169]{beamer}
\usetheme{metropolis}
\title{DAPC 2023 Training Sessions\\Session 4}
\author{Verwoerd}
\date{September 21, 2003}

% Packages
\usepackage{amsmath}
\usepackage[utf8]{inputenc}
\usepackage[T1]{fontenc}
\usepackage{graphicx}
\usepackage{tikz}
\usepackage{minted}
\usepackage[
  type={CC},
  modifier={by-sa},
  version={4.0},
]{doclicense}
\setsansfont{Fira Sans}
\usemintedstyle{manni}
\setminted{
  fontsize=\footnotesize,linenos,frame=lines, framesep=2mm
}
\usetikzlibrary{angles,quotes,graphs, graphdrawing,positioning}
\usegdlibrary {layered}

\begin{document}
  \maketitle
  \begin{frame}{Session 4}
    \begin{itemize}
      \item Role of the coach on big contests
      \item Tips, tricks and common mistakes
      \item Solutions to the Interactive Problems and Dynamic Programming Problems
      \item Dealing with randomization
      \item Solutions the hardest problems
    \end{itemize}
    \doclicenseThis
  \end{frame}


  \section{Role of the coach}


  \section{Tips, tricks and common mistakes}


  \section{Solutions to the Interactive Problems and Dynamic Programming Problems}
  \begin{frame}{Cookbook Composition}
    \begin{itemize}
      \item Source BAPC Preliminaries 2022
      \item Time limit: 2s
      \item Given a list of recipes, print the order the recipes by accessibility (lowest $\frac{beginner\text{ }time}{expert\text{ }time}$ first).

    \end{itemize}
    Original problem written by the BAPC 2022 jury and licensed under \doclicenseLongNameRef.

    \doclicenseImage
  \end{frame}

  \begin{frame}{Cookbook Composition}
    \begin{itemize}
      \item Observation: $n\cdot s \leq 2.5\cdot10^4$, so we are aiming for a $\mathcal{O}(n\cdot s \log{n\cdot s})$
      \item Observation: Steps are in order, dependencies are always declared first
      \item The beginner time is trivial to calculate, the sum of the time of all steps
      \item The expert time can be defined as a DP relation\\
      \[\text{time}_s = \begin{cases}
                          t & \text{if no dependencies are given}\\
                          t + \max\limits_{steps}^{i=0} time_i & \\
      \end{cases}\]
      \item This can be calculated in linear time by processing the steps one by one, where the expert time is the time of the last step.
      \item Then sort the recipes based on $\frac{beginner\text{ }time}{expert\text{ }time}$ and print out the result
      \item This results in a $\mathcal{O}(n\cdot s + n \log n)$ solution
    \end{itemize}
  \end{frame}
  \begin{frame}[containsverbatim]{Cookbook Composition}
    \inputminted{python}{code/session-4/dapc-c.py}
  \end{frame}
  \begin{frame}{Guessing Primes}
    \begin{itemize}
      \item Source BAPC Preliminaries 2022
      \item Interactive Problem
      \item Time limit: 10s
      \item Guess the hidden 5-digit prime in at most 6 guesses, i.e., play Primel.

    \end{itemize}
    Original problem written by the BAPC 2022 jury and licensed under \doclicenseLongNameRef.

    \doclicenseImage
  \end{frame}
  \begin{frame}{Guessing Primes}
    \begin{itemize}
      \item There are 8363 primes between 10000 and 99999, can be generated within the timelimit of 10s
      \item Primality can be checked an odd number $n$ can not be divided by $(3..\lfloor\sqrt{n}\rfloor)$
      \item Selecting a random prime and use the rules to generate the next guess will take on average 7 guesses.
      \item The reason is when only one digit is known, to many guesses needed for the other 4
      \item So your first 2 guesses should all contain different digits, like 24683 and 10597
      \item Use the digits to generate the next guesses
      \item This is guaranteed you can do this in 6 guesses
    \end{itemize}
  \end{frame}
  \begin{frame}[containsverbatim]{Guessing Primes(1)}
    \inputminted[fontsize=\tiny]{python}{code/session-4/dapc-g-1.py}
  \end{frame}
  \begin{frame}[containsverbatim]{Guessing Primes(2)}
    \inputminted[fontsize=\tiny]{python}{code/session-4/dapc-g-2.py}
  \end{frame}
  \begin{frame}{Dividing DNA}
    \begin{itemize}
      \item Source BAPC 2022
      \item Interactive Problem
      \item Time limit: 2s
      \item  Given a set of forbidden (present) intervals, partition $[0,n)$ into as many disjoint (absent) intervals as possible with at most $2n$ queries.

    \end{itemize}
    Original problem written by the BAPC 2022 jury and licensed under \doclicenseLongNameRef.

    \doclicenseImage
  \end{frame}
  \begin{frame}{Dividing DNA}
    \begin{itemize}
      \item If an interval is forbidden, then all shorter intervals are forbidden too.
      \item An absent interval is just one longer then a forbidden interval.
      \item A greedy solution works here
      \item start with $[0,1)$ and keep growing until an absent interval is found
      \item Then start at the last exclusive boundary a new boundary and continue until the end is reached.
      \item \includegraphics[width=.8\linewidth]{images/session-4/bapc-d}
      \item The result is the number of intervals with $n$ queries
    \end{itemize}

  \end{frame}
  \begin{frame}[containsverbatim]{Dividing DNA}
    \inputminted{python}{code/session-4/bapc-d.py}
  \end{frame}

  \begin{frame}{Jaged skylines}
    \begin{itemize}
      \item Source BAPC 2022
      \item Interactive Problem
      \item Time limit: 4s
      \item  Given $w \leq 10000$ integers $0 \leq h_i \leq 10^{18}$ , find the maximum in at most 12000 queries: ``Is integer $h_i$ less than y?''
    \end{itemize}
    Original problem written by the BAPC 2022 jury and licensed under \doclicenseLongNameRef.

    \doclicenseImage
  \end{frame}
  \begin{frame}{Jaged skylines}
    \begin{itemize}
      \item For every column we can binary search the highest tile
      \item This is 5000 queries, so too much
      \item Rather than start in the middle, we can check if it is higher than the best found
      \item and then binary search only found
      \item Worst case: the maximum increases with every column
      \item Randomize the order, the change that an item is higher is $\ln(w)$
      \item Resulting in number of queries of $w+\ln(w)\cdot \log(h)$
      \item Note that this a Monte Carlo estimate, which was manually monitored by the jury
    \end{itemize}
  \end{frame}
  \begin{frame}[containsverbatim]{Dividing DNA}
    \inputminted{python}{code/session-4/bapc-j.py}
  \end{frame}

  \section{Dealing with randomization}


  \section{Solving the Hardest Problems}


  \section{Guest Speaker}


  \section{Conclussion}

\end{document}
