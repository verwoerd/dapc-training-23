% Preamble
\documentclass[11pt,pdf, aspectratio=169]{beamer}
\usetheme{metropolis}
\title{DAPC 2023 Training Sessions\\Session 2}
\author{Verwoerd}

% Packages
\usepackage{amsmath}
\usepackage[utf8]{inputenc}
\usepackage[T1]{fontenc}
\usepackage{graphicx}
\usepackage{tikz}
\usepackage{minted}
\usepackage[
  type={CC},
  modifier={by-sa},
  version={4.0},
]{doclicense}
\setsansfont{Fira Sans}
\usemintedstyle{manni}
\setminted{
  fontsize=\footnotesize,linenos,frame=lines, framesep=2mm
}

\begin{document}
  \maketitle


  \section{Solutions to the ad-hoc and math problems}
  \begin{frame}{Extended Braille}
    \begin{itemize}
      \item Source BAPC Preliminaries 2022
      \item Time limit: 8s
      \item Given $n$ braille characters by their points, determine how many of them are distinct up to translation.
    \end{itemize}
    Original problem written by the BAPC 2022 jury and licensed under \doclicenseLongNameRef.

    \doclicenseImage

  \end{frame}
  \begin{frame}{Extended Braille}
    \begin{itemize}
      \item Observation 1: time limit of 8s is due to high input size
      \item Observation 2: Number of inputs is $10^6$, so we are looking for a $\mathcal{O}(n\log{}n)$
      \item Per Braille character sort the dots on $x$ then $y$
      \item Move the first ordered dot to $(0, 0)$ by subtracting the first point coordinate from all the dots\\
      $\all^{i=0}_m (x'_i, y'_1) = (x_i - x_o, y_i - y_0)$
      \item Add the transposed characters to a Hashmap and count the unique keys
      \item Resulting a $\mathcal{O}(n\log{}n)$ or amortized $\mathcal{O}(n)$
    \end{itemize}
  \end{frame}
\end{document}
